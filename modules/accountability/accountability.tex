\emph{Summary:}
The need for accountability has been apparent in civilisations for as long as 
they have existed.
One of today's institutions which is historically renowned for keeping strict 
accounts is the state tax office, another is, of course, the banks.
We will explore some principles in keeping accounts and discuss ways to 
implement it in different, sometimes challenging, environments.
We will also see how these principles are applied in \acp{DLT}, such as 
Bitcoin.

\emph{Intended learning outcomes:}
In particular, the \acp{ILO} are that you are able to:
\begin{itemize}
  \item \emph{evaluate} advantages and disadvantages of different levels of 
    accountability.
  \item \emph{analyse} a situation and \emph{design} proper accountability and,
    in particular, with privacy considerations.
\end{itemize}

\emph{Reading:}
Anderson describes accountability through his experience from banks in Chapter 
10 \enquote{Banking and Bookkeeping} in 
\citetitle{Anderson2008sea}~\cite{Anderson2008sea}.

We will also use the secure logging system of 
\citeauthor{schneier1999secure}~\cite{schneier1999secure} as an example of how 
to achieve secure logging in a challenging environment.
The construction described therein is a method to safely store audit logs in an 
untrusted machine; in the scheme, all log entries generated prior to 
a compromise will be impossible for the attacker to read, modify, or destroy 
undetectably.
The core principle is a blockchain.

We will look into \acp{DLT} (blockchains).
This is covered by \textcite[Sect.~3, 4]{NISTblockchainOverview}.
