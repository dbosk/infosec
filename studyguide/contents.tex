\mode*

\section{Scope and aims}%
\label{sec:aim}

The aim of the course is that after the course you should be able to make 
high-level designs for secure solutions, \ie combine relevant research results 
based on their high-level properties into a solution with the desired security, 
privacy and usability properties.
The problems and solutions can be in both the technical or organizational 
domain.

\mode<presentation>{%
  \subsection{Scope}

  \begin{frame}
    \begin{itemize}
      \item We consider two general situations.
    \end{itemize}

    \begin{enumerate}
      \item You must acquire some system from a third party.
      \item You participate in the design of a new system.
    \end{enumerate}

    \begin{itemize}
      \item In both cases you must know the possibilities of security.
    \end{itemize}
  \end{frame}
}

\subsection{\Aclp*{ILO}}
\mode<all>{%
\input{ilo-grading.tex}
}


\section{Course structure and overview}%
\label{CourseStructure}

The course is divided into three parts.
The first part of the course covers the foundations of security: what it is, 
how to evaluate new knowledge in the field.
This covers both purely technical aspects, but also includes human aspects such
as usability --- even if a system is proved secure, it will offer no security 
if its human users cannot use it.

The second part of the course covers information security on a strategic level, 
this concerns organizational management systems for information security: how 
to implement these and how to continuously run them in an organization.
It also includes threat and risk analysis.
The main material is produced by the Swedish Civil Contingencies Agency (MSB) 
and is based on the ISO 27000 standard.

The third part of the course covers the technical aspects: how to design 
security (and not to design security).
The focus in this part of the course is on security mechanisms and how to use 
these in secure systems.

\subsection{Teaching and tutoring}

\mode<presentation>{%
\begin{frame}
  \begin{remark}
    \begin{itemize}
      \item The course uses flipped classroom and active learning.
      \item There are videos to watch and texts to read before sessions.
      \item Then you'll be more active during the sessions.
    \end{itemize}
  \end{remark}
\end{frame}
}

The teaching of the course is oriented towards active learning.
\Ie the course consists of learning sessions which requires active 
participation.

Each topic is covered by some recorded lectures and an interactive session.
For each topic the reading material is specified.
Generally, you are expected to watch the videos and read the material in 
advance.
During the (interactive) learning session the most important parts of the 
material will be discussed and you will perform some tasks to work with the 
topic in groups, \ie to apply it to learn it more efficiently.
Some modules of the course will have several learning sessions linked together,
\eg a starting seminar, followed by laboratory work which is then summarized 
and used in a final seminar.

\subsection{Schedule}

In \cref{Schedule} you will find an overview of the schedule for the course.
The detailed schedule can be found in the University's central scheduling 
system.
An abstract for each session can be found in Section~\ref{CourseContents} and 
the details in the actual material.

\begin{frame}[allowframebreaks]
\mode<article>{%
\begin{table}
}
	\centering
  \begin{tabular}{lp{9cm}}
    \toprule
    \textbf{Course week}	& \textbf{Work} \\
    \midrule
    1
      & Session: Introduction\\
      & Seminar: What's up with security?
        (Section~\ref{security-society-seminar})\\
      & Session: Foundations
        (Section~\ref{foundations})\\
    \midrule
    2
      & Lecture: MSB's framework, part I
        (Section~\ref{msb-framework})\\
      & Lecture: MSB's framework, part II\\
      & Lecture: Records management\\
    \midrule
\mode<presentation>{%
  \end{tabular}
  \begin{tabular}{lp{9cm}}
}%
    3
      & Session: Crypto I
        (Section~\ref{crypto})\\
      & Session: Crypto II
        (Section~\ref{crypto})\\
      & First grading: M1 (isms), M2 (risk)\\
    \midrule
    4
      & Session: Authentication
        (Section~\ref{authentication})\\
      & Seminar: L4 Evaluating and Designing Authentication
        (Section~\ref{pwdeval})
        part I\\
    \midrule
\mode<presentation>{%
  \end{tabular}
  \begin{tabular}{lp{9cm}}
}%
    5
      & Seminar: L4 part II\\
      & Seminar: L5 Private Communication
        (Section~\ref{pricomlab})\\
      & Session: Protocols
        (Section~\ref{protocols})\\
    \midrule
    6
      & Session: Access control
        (Section~\ref{ac})\\
      & Session: Accountability
        (Section~\ref{accountability})\\
      & Session: Differential privacy
        (Section~\ref{diffpriv})\\
    \midrule
\mode<presentation>{%
  \end{tabular}
  \begin{tabular}{lp{9cm}}
}%
    7
      & Session: Software security
        (Section~\ref{software})\\
      & Session: Trusted computing
        (Section~\ref{trustcomp})\\
      & Seminar: S3
        (Section~\ref{risksem})\\
      & Session: Conclusion
        (Section~\ref{conclusion})\\
    \midrule
    8
      & Tutoring: P6
        (Section~\ref{devel})\\
    \midrule
    9
      & Tutoring: P6 (devel)\\
    \midrule
    10
      & Presentation: P6 (devel)\\
      & Second grading: M1 (isms), M2 (risk)\\
      & Second seminar: S3 (risk), L4 (pwdeval), L5 (pricomlab)\\
    \midrule
\mode<presentation>{%
  \end{tabular}
  \begin{tabular}{lp{9cm}}
}%
    +3 months
      & Second presentation: P6 (devel)\\
      & Final grading: M1 (isms), M2 (risk)\\
      & Final seminar: S3 (risk), L4 (pwdeval), L5 (pricomlab)\\
    \midrule
    +6 months
      & Final presentation: P6 (devel)\\
    \bottomrule
  \end{tabular}
\mode<article>{%
  \caption{%
    A summary of the parts of the course and when they will (or should) be done.
    The table is adapted to taking this course at half-time pace, \ie 20 hours 
    per week for 10 weeks.
  }\label{Schedule}
\end{table}
}


\end{frame}


\section{Course contents}%
\label{CourseContents}

This section summarizes each of the learning sessions, \ie what they cover, 
what you are expected to learn and its reading material.

\mode<article>{%
%\subsection{L0 Privacy is Dead}
%\input{privacydead-abstract.tex}
%
\subsection{S0 What's up with security?}%
\label{security-society-seminar}
\input{intro-abstract.tex}

\subsection{Foundations}%
\label{foundations}
\input{foundations-abstract.tex}

\subsection{Managing information security}%
\label{msb-framework}

\subsubsection{MSB part I}
\input{msbintro-abstract.tex}

\subsubsection{M1 Information security management system}%
\label{ism}
\input{ismsmemo-abstract.tex}

\subsubsection{MSB part II}
\input{msbforts-abstract.tex}

\subsubsection{M2 and S3 Assessment and risk analysis}%
\label{risksem}
\input{risksem-abstract.tex}

\subsubsection{Information security from a records management perspective}
\input{recmgmt-abstract.tex}

\subsection{Cryptography}%
\label{crypto}
\input{crypto-abstract.tex}

\subsection{Authentication}%
\label{authentication}
\input{auth-abstract.tex}

\subsubsection{L4 Evaluating and designing authentication}%
\label{pwdeval}
\input{pwdeval-abstract.tex}

\subsection{Protocols}%
\label{protocols}
\input{proto-abstract.tex}

\subsubsection{L5 Private communication}%
\label{pricomlab}
\input{pricomlab-abstract.tex}

\subsection{Access control}%
\label{ac}
\input{accessctrl-abstract.tex}

\subsection{Accountability}%
\label{accountability}
\input{accountability-abstract.tex}

\subsection{Differential privacy}%
\label{diffpriv}
\input{diffpriv-abstract.tex}

\subsection{Software security}%
\label{software}
\input{software-abstract.tex}

\subsection{Trusted computing}%
\label{trustcomp}
\input{trustcomp-abstract.tex}

\subsection{Conclusion}%
\label{conclusion}

We will finish the course with a summarizing session and then start the 
project.

%\subsubsection{P8 A short study in information security}
\subsubsection{P6 Applying security and usability in practice}%
\label{devel}
\input{project-abstract.tex}
} % mode<article>


\section{Assessment}%
\label{Assessment}
\mode<all>{\input{ladok.tex}}

\subsection{Handed-in assignments}
\mode<all>{\input{hand-ins.tex}}

\subsection{\enquote{What if I'm not done in time?}}%
\label{sec:late}
\mode<all>{\input{late.tex}}


\printbibliography
